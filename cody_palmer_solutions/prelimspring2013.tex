\documentclass{article}
\usepackage{amssymb, latexsym, amsmath, eucal, graphics, fullpage, epsfig, amsthm}
\newtheorem{theorem}{Theorem}
\newtheorem{lemma}{Lemma}
\newtheorem{proposition}{Proposition}
\newtheorem{definition}{Definition}
\newtheorem{corollary}{Corollary}
\newtheorem{notation}{Notation}

\begin{document}
\setlength{\baselineskip}{18pt}
\textbf{1.} Let $(a_n)$ and $(b_n)$ be sequences of positive numbers such that
\[
	\inf_{n \in \mathbb{N}} \left( b_n - b_{n+1}\frac{a_{n+1}}{a_n} \right) > 0.
\]
Prove that $\sum_{n=1}^\infty a_n < \infty$.
\vskip.1in
\textbf{Solution}.  We shall prove this by considering the sequence of partial sums
\[
	s_k = \sum_{n=1}^k a_n
\]
and showing that this is a monotonically increasing sequence that is bounded above, and must therefore be convergent.  That it is monotonically increasing is obvious since the terms $a_i$ are positive.  So it only remains to be  shown that it is bounded above.  Since we have
\[
\inf_{n \in \mathbb{N}} \left( b_n - b_{n+1}\frac{a_{n+1}}{a_n} \right) > 0.
\]
we get that there is an $\alpha>0$ such that for all $n$ 
\[
	b_n - b_{n+1}\frac{a_{n+1}}{a_n} \geq \alpha \Rightarrow a_nb_n - a_{n+1}b_{n+1} \geq \alpha a_n.
\]
Since this is true for all $n$, we can sum both sides over $n=1$ to $n=k$ to get that
\[
	\sum_{n=1}^k a_nb_n - \sum_{n=1}^k a_{n+1}b_{n+1} \geq \alpha s_k
\]
\[
	\Rightarrow a_1b_1 + \sum_{n=2}^k a_nb_n - \sum_{n=2}^k a_{n}b_{n} + a_{k+1}b_{k+1} \geq \alpha s_k
\]
\[
	\Rightarrow \alpha s_k \leq a_1b_1-a_{k+1}b_{k+1} < a_1b_1
\]
So then the sequence of partial sums is bounded above by $\frac{1}{\alpha}a_1b_1$, and being monotonically increasing, must be convergent.  Hence $\sum a_n < \infty$.
\vskip.1in
\textbf{Problem 2.} Let $f$ be a continuous real valued function on $[0,1]$, which is twice differentiable and satisfies $f(0) = f(1)$.  Suppose that $M > 0$ is such that $|f^{\prime \prime}(x)|\leq M$ for all $0<x<1$. Prove that $|f^\prime (\frac{1}{2})| \leq \frac{M}{4}$ and $|f^\prime 
(x)| \leq \frac{M}{2}$, for all $0<x<1$.
\vskip.1in
\textbf{Solution}.  We start by making the following observation for any $y \in (0,1)$:
\[
	\left|\int_0^y \int_x^y f^{\prime \prime}(t) dtdx - \int_y^1 \int_y^x f^{\prime \prime}(t)dtdx\right|=|f^\prime(y)|
\]
This is seen to be true by just applying the fundamental theorem of calculus:
\[
	\int_0^y \int_x^y f^{\prime \prime}(t) dtdx - \int_y^1 \int_y^x f^{\prime \prime}(t)dtdx = \int_0^y f^\prime(y) - f^\prime(x) dx - \int_y^1 f^\prime(x) - f^\prime(y)dx
\]
\[
	= yf^\prime(y) - (f(y) - f(0))+ \int_y^1 f^\prime(y) - f^\prime(x) dx = yf^\prime(y) - f(y) + f(0) + (1-y)f^\prime(y) - (f(1) - f(y))
\]
\[
	= f^\prime(y) - f(y) + f(0) - f(1) + f(y) = f^\prime(y) \mbox{ since } f(0)=f(1)
\]
Now since $|f^{\prime \prime}(x)| \leq M$, we get that
\[
	|f^\prime(y)| = \left|\int_0^y \int_x^y f^{\prime \prime}(t) dtdx - \int_y^1 \int_y^x f^{\prime \prime}(t)dtdx\right| \leq \int_0^y \int_x^y |f^{\prime \prime}(t)| dtdx + \int_y^1 \int_y^x |f^{\prime \prime}(t)|dtdx
\]
\[
	\leq \int_0^y \int_x^y M dtdx + \int_y^1 \int_y^x M dtdx = \int_0^y M(y-x) dx + \int_y^1 M(x-y)dx = M \left( \left. \left( yx - \frac{x^2}{2}\right)\right|_0^y + \left. \left( \frac{x^2}{2}-yx\right)\right|_y^1 \right)
\] 
\[
	= M \left( y^2 - \frac{y^2}{2} + \frac{1}{2} - y - \frac{y^2}{2} +y^2 \right) = M\left(y^2-y+\frac{1}{2} \right)
\]
Now it is a matter of basic algebra to see that $g(y) = y^2-y+\frac{1}{2}$ is a parabola with minimum $g(\frac{1}{2}) = \frac{1}{4}$, and $g(0) = g(1) = \frac{1}{2}$ as the maximums over the unit interval.  This gives that 
\[
	\left|f^\prime\left(\frac{1}{2}\right)\right| \leq \frac{M}{4} \mbox{ and } |f^\prime(x)| \leq \frac{M}{2}
\]
\vskip.1in
\textbf{Problem 3.} Let $f$ be a continuous real valued function on $[0,1]$.  Find the following limits
\[
	\mbox{(a) } \lim_{n \to \infty} n \int_0^1 x^nf(x) dx.
\]
\[
	\mbox{(b) } \lim_{n \to \infty} n \int_0^1 e^{-nx} f(x) dx.
\]
\vskip.1in
\textbf{Solution. } We claim that the limit in (a) is $f(1)$, and the limit in (b) is $f(0)$.  We first show that if the result is true for polynomials, then it will hold for all continuous functions.  To do this we will use the Stone-Weierstrass Theorem.  We can use this theorem because we are looking at continuous real valued functions over a compact set $[0,1]$.  Let $p_n$ be a sequence of polynomials that converge uniformly to $f$.  Then, since we have unform convergence of thei sequence of functions, then we have that it can be integrated term by term.  This gives that
\[
	 \lim_{n \to \infty} n \int_0^1 x^nf(x) dx  =  \lim_{n \to \infty} n \int_0^1 x^n\lim_{m \to \infty}p_m(x) dx = \lim_{m \to \infty}\lim_{n \to \infty}  n \int_0^1 x^np_m(x)
\]
\[
	 = \lim_{m \to \infty} p_m(1) = f(1)
\]
Similarly
\[
	 \lim_{n \to \infty} n \int_0^1 e^{-nx}f(x) dx  =  \lim_{n \to \infty} n \int_0^1 e^{-nx}\lim_{m \to \infty}p_m(x) dx = \lim_{m \to \infty}\lim_{n \to \infty}  n \int_0^1e^{-nx}p_m(x)
\]
\[
	 = \lim_{m \to \infty} p_m(0) = f(0)
\]
So then, we need show the result for polynomials.  We shall now consider the first limit.  Let $p(x) = \sum_{k=0}^m a_kx^k$ be a polynomial.  We have that
\[
	 \lim_{n \to \infty} n \int_0^1 x^np(x) dx = \lim_{n \to \infty} n \sum_{k=0}^m a_k \int_0^1 x^{n+k} dx =  
\]
\[
	\lim_{n \to \infty} n \sum_{k=0}^m a_k \left.\frac{x^{n+k+1}}{n+k+1}\right|_0^1 =    \sum_{k=0}^m a_k \lim_{n \to \infty} \frac{n}{n+k+1} = \sum_{k=0}^m a_k = p(1)
\]
So then we have established the first limit.\\
For the second limit, we shall first establish the following by induction
\[
	\lim_{n \to \infty} n \int_0^1 e^{-nx}x^m dx = 0 \mbox{ for } m>0
\]
 For $m=1$, we use integration by parts to get that
\[
	\lim_{n \to \infty} n \int_0^1 e^{-nx}x dx = \lim_{n\to \infty}n\left(\left. -\frac{xe^{-nx}}{n}\right|_0^1  + \frac{1}{n} \int_0^1 e^{-nx}dx \right) = \lim_{n \to \infty} n \left( \frac{-e^{-n}}{n} + \frac{1}{n} \left(-\frac{e^{-n}}{n} + \frac{1}{n}\right)\right)
\] 
\[
	 = \lim_{n \to \infty} -e^{-n} + \frac{1}{n} - \frac{e^{-n}}{n} = 0
\]
Now we assume true for $m < k$, and consider the case $m=k$. Again applying integration by parts we get
\[
	\lim_{n \to \infty} n \int_0^1 e^{-nx}x^k dx  = \lim_{n \to \infty} n \left( \left. -\frac{e^{-nx}x^k}{n} \right|_0^1 + \frac{1}{n} \int_0^1 ne^{-nx}x^{k-1}dx \right)
\]
\[
	= \lim_{n \to \infty} e^{-n} +\lim_{n \to \infty} n \int_0^1 e^{-nx}x^{k-1} dx = 0+0= 0
\]
Thus, we have the result for all $m>0$.  For $m=0$ we see that
\[
	\lim_{n \to \infty} n\int_0^1 e^{-nx} = \lim_{n \to \infty} n\left(-\frac{e^{-n}}{n} + \frac{1}{n}\right) = \lim_{n \to \infty} -e^{-n} +1 = 1
\]
So, for an arbitrary polynomial $p(x)$ we have that
\[
	\lim_{n \to \infty}  n \int_0^1e^{-nx}p(x) = \sum_{k=0}^m a_k \lim_{n \to \infty} n \int_0^1 e^{-nx}x^k dx = a_0 + 0 + \ldots + 0 = a_0 = p(0)
\]
So then we have established the second limit for polynomials, and thus all continuous functions.
\vskip.1in
textbf{Problem 5.}
\vskip.1in
\textbf{Solution}
\vskip.1in
\textbf{Problem 6.}
\vskip.1in
\textbf{Solution}
First we prove the hint.  We shall do so by induction.  For the case $m=2$ the right hand side of the formula becomes (using the observation that for $i < j$, $r_i - r_j = \sum_{k=i}^{j-1} a_k$),
\[
        r_1 - \frac{r_3}{2} - \frac{r_2}{2} = r_1 - \frac{r_3 +r_2}{2} = r_1 - \frac{2r_3 + a_2}{2}
\] 
\[
	= r_1 - r_3 - \frac{a_2}{2} = a_1 + a_2 - \frac{a_2}{2} = a_1 + \frac{a_2}{2}
\]
which is equal to the left hand side of the formula for $m=2$.  Now assume it is the case for $m=p$.  Then we have that
\[
	\sum_{k=1}^p \frac{a_k}{k} = r_1-\frac{r_{p+1}}{p}-\sum_{k=2}^p \frac{r_k}{k(k-1)}
\]
Adding $\frac{a_{p+1}}{p+1}$ to both sides gives that
\[
	\sum_{k=1}^{p+1} \frac{a_k}{k} = r_1-\frac{r_{p+1}}{p}-\sum_{k=2}^p \frac{r_k}{k(k-1)} + \frac{a_{p+1}}{p+1}
\]
\[
	= r_1-\frac{r_{p+1}}{p}-\sum_{k=2}^p \frac{r_k}{k(k-1)} + \frac{r_{p+1} - r_{p+2}}{p+1}
\]
\[
	= r_1 - \frac{r_{p+2}}{p+1} - \sum_{k=2}^p \frac{r_k}{k(k-1)} -\frac{r_{p+1}}{p} + \frac{r_{p+1}}{p+1}
\]
\[
	= r_1 - \frac{r_{p+2}}{p+1} - \sum_{k=2}^p \frac{r_k}{k(k-1)}  +\frac{r_{p+1}}{p(p+1)}
\]
\[
	= r_1 - \frac{r_{p+2}}{p+1} - \sum_{k=2}^{p+1} \frac{r_k}{k(k-1)}  
\]
And this it is true for $m=p+1$, and we have establsihed the result by induction.\\
Now, in order to prrove part (a), we will use this formula for the sequence of partial sums and show that it converges.  First notice that since $r_n$ is the tail of a coverging series, we must have that $r_n \to 0$.  So then there is an $M$ such that for all $n \geq M$, we get $|r_n| < 1$.  So then we have that for all $n \geq M$
\[
	\frac{|r_k|}{k(k-1)} < \frac{1}{k(k-1)}
\]
and since the series $\sum \frac{1}{k(k-1)}$ converges (expand using partial fractions and partial sums, and it telescopes nicely), we must have that
\[
	\sum_{k=2}^\infty \frac{r_k}{k(k-1)} \mbox{ converges.}
\]
As a result, we directly apply the formula for the partial sums above to get that
\[
	\lim_{n \to \infty} \sum_{k=1}^n \frac{a_k}{k} = r_1 - \sum_{k=2}^\infty \frac{r_k}{k(k-1)} \mbox{ converges.}
\]
Thus we have shown part (a).
\vskip.1in
\textbf{Problem 7}
\vskip.1in
\textbf{Solution} Let $\epsilon > 0$ be given and let $N$ be such that $|x_n| \leq \frac{\epsilon}{2K}$.  Now applying the triangle inequality we have that, for $n \geq N$:
\[
	|a_{n1}{x_1} + \ldots +a_{nn}x_n| \leq | a_{n1}x_1 + \ldots +a_{nN}x_{N}| + |a_{nN+1}x_{N+1} + \ldots +a_{nn}x_n| 
\]
\[
	\leq | a_{n1}x_1 + \ldots +a_{nN}x_{N}| +|a_{nN+1}||x_{N+1}| + \ldots + |a_{nn}||x_n|
\]
\[
	< | a_{n1}x_1 + \ldots +a_{nN}x_{N}|  + \frac{\epsilon}{2K}(|a_{nN+1}| + \ldots + |a_{nn}|)
\]
\[
	\leq | a_{n1}x_1 + \ldots +a_{nN}x_{N}|  + \frac{\epsilon}{2K}(|a_{n1}| + \ldots+ |a_{nn}|) \leq | a_{n1}x_1 + \ldots +a_{nN}x_{N}|  + \frac{\epsilon}{2K} K
\]
\[
	 = | a_{n1}x_1 + \ldots +a_{nN}x_{N}|  + \frac{\epsilon}{2}
\]
Now each of the terms in the first sum individually converge to 0, and thus the finite sum must converge to 0, hence there is an $N^\prime > N$ such that $| a_{n1}x_1 + \ldots +a_{nN}x_{N}| <\frac{\epsilon}{2}$ for all $n \geq N^\prime$, and hence
\[
	|a_{n1}x_1 + \ldots +a_{nn}x_n| < \frac{\epsilon}{2} + \frac{\epsilon}{2} = \epsilon
\]
for all $n \geq N^\prime$.  Thus
\[
	a_{n1}x_1 + \ldots +a_{nn}x_n \to 0
\]
\vskip.1in
\textbf{Problem 8.}
\vskip.1in
\textbf{Solution} We shall first show that $f$ is uniformly continuous and thus continuous.  First note that since $X$ is compact, then we must have that $f_n$ is uniformly equicontinuous.   this follows from noting that the negation of uniform equicontinuity is that there is one $f_n$ is the sequence that fails to be uniformly continuous, but since $X$ compact, and each $f_n$ is continuous, they all must be uniformly continuous.  Now then we can go on to show the uniform continuity of $f$.  Let $\epsilon<0$ be given, and let $\delta$ be such that $d(x,y) < \delta$ implies $|f_n(x) - f_n(y)| < \frac{\epsilon}{2}$ for all $n$ (we appeal to the uniform equicontinuity to get such a $\delta$).  Now let $n \to \infty$, and since we have pointwise convergence, and absolute value is continuous, we get that
\[
	|f(x) - f(y)| = \lim_{n \to \infty}|f_n(x) - f_n(y)| \leq \frac{\epsilon}{2} < \epsilon
\]
Hence $f$ is uniformly continuous.\\
We can now move on to proving that the convergence is uniform.  let $\epsilon > 0$ be given, and suppose that $\delta_1$ is such that $d(x,y) < \delta_1 \Rightarrow |f_n(x) - f_n(y)| < \frac{\epsilon}{3}$, we are appealing to the uniform equicontinuity to  get $\delta_1$.  Similarly, let $\delta_2$ be such that $d(x,y) < \delta_2 \Rightarrow |f(x) -f(y)| < \frac{\epsilon}{3}$, and we are appealing to the uniform continuity of $f$ to get such a $\delta_2$.  Now let $\delta = \min \{\delta_1, \delta_2\}$.  We note that 
\[
	\bigcup_{x \in X} B(x,\delta)
\]
forms an open cover of $X$, and by compactness, we must have a finite subcover i.e. there are finitely many $x_i \in X$ such that
\[
	X = \bigcup_{i=1}^k B(x_i,\delta).
\]
Now let $N_i$ be such that $|f_n(x_i) - f(x_i)| < \frac{\epsilon}{3}$ for all $n \geq N_i$, and let $N = \max_{1 \leq i \leq k} N_i$.  Now, for any $x$ there is an $x_i$ such that $x \in B(x_i,\delta)$, and thus for all $n \geq N$, we have, forr any $x \in X$:
\[
	|f_n(x) - f(x)| =|f_n(x) - f_n(x_i) + f)n(x_i) - f(x_i) + f(x_i) - f(x)| \leq |f_n(x) - f_n(x_i)| + |f_n(x_i) - f(x_i)| + |f(x_i) - f(x)|
\]
Since $d(x,x_i) <\delta < \delta_1$ we get that $ |f_n(x) - f_n(x_i)| < \frac{\epsilon}{3}$, and similarly, since $d(x,x_i) < \delta < \delta_2$ we get that $|f(x_i) - f(x)| < \frac{\epsilon}{3}$.  Because $n \geq N \geq N_i$ we have that $|f_n(x) - f_n(x_i)| <\frac{\epsilon}{3}$.  Combining all these we get that for all $n \geq N$ and any $x \in X$:
\[
	|f_n(x) - f(x)| < \frac{\epsilon}{3} + \frac{\epsilon}{3} + \frac{\epsilon}{3} = \epsilon
\]
So then $f_n$ converges uniformly to $f$.
\vskip.1in
\textbf{Problem 9}.
\vskip.1in
\textbf{Solution}
First we will show that $g$ is continuous.  Forst note that the $d(x, A)$ are continuous, following from the continuity of metrics.  So the only thing required to show that $g$ is continuous is to show that $d(X,A) +d(x,B) > 0$.  Since $d(x,A),d(x,B) \geq$ we need only show that it is impossible to have that $d(x,A)+d(x,B) = 0$.  Assume that this is the case for some value $x \in X$, then we must have that $d(x,A) = d(x,B) = 0$.  Since, in general, $d(x,S) = \inf \{d(x,y) : y \in S \}$, we have a sequence $y_n \subset A$ and $z_n \subset B$ such that $d(x,y_n) \to 0$ and $d(x,z_n) \to 0$.  Since $f$ is continuous, we have that $d(x,y_n) \to 0 \Rightarrow |f(x) - f(y_n)| \to 0$, and similarly $d(x,z_n) \to 0 \Rightarrow |f(z_n) - f(x)| \to 0$.  Using the triangle inequality we get that
\[
	|f(z_n)-f(y_n)| = |f(z_n)-f(x) + f(x)-f(y_n)| \leq |f(z_n)-f(x)|+|f(x)-f(y_n)| \to 0
\]
and thus $|f(z_n) - f(y_n)| \to 0$.  now since $z_n \subset B$ and $y_n \subset A$ we have that $f(z_n) \geq \frac{M}{3}$ and $f(y_n) \leq \frac{-M}{3}$ for all $n$.  Thus
\[
	|f(z_n) - f(y_n)| = f(z_n) - f(y_n) \geq \frac{2M}{3} \mbox{ for all } n
\]
which contradicts that $|f(z_n)-f(y_n)| \to 0$.  So then $d(x,A)+d(x,B) > 0$ and $g$ is continuous.  Furthermore
\[
	|g(x)| = \frac{M}{3}\frac{|d(x,A) - d(x,B)|}{d(x,A)+d(x,B)} \leq \frac{M}{3} \frac{d(x,A)+d(x,B)}{d(x,A)+d(x,B)} = \frac{M}{3}
\]
Now for $y \in A$ it is easy to see that $g(y) = -\frac{M}{3}$, and so
\[
	 -M \leq f(y) \leq -\frac{M}{3} \Rightarrow -\frac{2M}{3} \leq f(y) +\frac{M}{3} \leq 0 \Rightarrow -\frac{2M}{3} \leq f(y) - g(y) \leq 0
\]
and thus $|f(y) - g(y)| \leq \frac{2M}{3}$for all $y \in A$.  Similiarly, if $y \in B$ then we have that $g(y) = \frac{M}{3}$, and so
\[
	\frac{M}{3} \leq f(y) \leq M \Rightarrow 0 \leq f(y) - \frac{M}{3} \leq \frac{2M}{3} \Rightarrow 0 \leq f(y) - g(y) \leq \frac{2M}{3}
\]
and thus $|f(y) -g(y)| \leq \frac{2M}{3}$ for all $y \in B$.  Now if $y \notin A \cup B$, then we have that $|f(y)|\leq \frac{M}{3}$, and since we have already shown that $|g(y)|\leq \frac{M}{3}$, we get that
\[
	|f(y) - g(y)| \leq |f(y)| + |g(y)| \leq \frac{2M}{3}
\]
So, for any $y \in Y$ we have that $|f(y) - g(y)| \leq \frac{2M}{3}$.\\
Now, to show that second part.  Since $|f| \leq 1$, we know by the previous part that there is a function $h_1:X \to [-1,1]$ such that $|h_1| \leq \frac{1}{3}$, and that $|f(y)-h_1(y)| \leq \frac{2}{3}$.  Now then $f-h_1$ is a function on $Y$ that is bounded by $\frac{2}{3}$, so we apply the previous part to $f-h_1$ to get that there is an $h_2:X \to [-1,1]$ such that 
\[
	|h_2| \leq \frac{2}{9} \mbox{ and } |f(y)-h_1(y)-h_2(y)| \leq \frac{4}{9}. 
\]
We can continue to define $h_k$ by induction.  Let the $h_i:X \to [-1,1]$ be so defined for $1 \leq i \leq k$.  That is we have that 
\[
	|h_i| \leq \frac{2^{\i-1}}{3^i}\mbox{ and that } |f(y) - h_1(y) - \ldots - h_i(y)| \leq \frac{2^i}{3^i}, \mbox{ for all } y \in Y \mbox{ and } 1\leq i \leq k.
\]
Now we apply the first part to get an $h_{k+1} : X \to [-1,1]$ such that $|h_{k+1}| \leq \frac{2^k}{3^{k+1}}$ and that
\[
	 |f(y) - h_1(y) - \ldots - h_k(y) - h_{k+1}(y)| \leq \frac{2^{k+1}}{3^{k+1}}
\]
Now we claim that $g(x) = \sum_{i=1}^\infty h_i(x)$ is the desired function.\\
To show this, we must show two things: That $g$ is continuous, and $g(y) = f(y)$.  The latter is the easiest, noting, that by definition of $g$, for all $y \in Y$:
\[
	|f(y) - g(y)| = \lim_{n \to \infty} |f(y) - h_1(y) - \ldots - h_n(y)| \leq \lim_{n \to \infty} \left( \frac{2}{3}\right)^k = 0
\]
To show that $g$ is continuous, it is sufficient to show that the infinite sum of continuous function $\sum h_i$ converges uniformly.  To show this we appeal to the Weierstrass $M$-test:  Note that because 
\[
	|h_k| \leq \left(\frac{2^{k-1}}{3^k}\right) \mbox{ and } \sum_{k=1}^\infty \left(\frac{2^{k-1}}{3^k}\right)  = \frac{1}{3} \sum_{k=0}^\infty \left(\frac{2}{3}\right)^k = 1
\]
So then all the conditions of The Weierstrss $M$-test are fufilled, and we get that $\sum h_i$ is uniformly convergent, and thus being the uniformly convergent sum of continuous functions, $g$ must be continuous.
\end{document}
